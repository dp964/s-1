\documentclass[11pt,a4paper]{article}
\usepackage[utf8]{inputenc}
\usepackage{amsmath,amsthm}
\usepackage{amssymb}
\usepackage[english]{babel}
\usepackage{changepage} 
\usepackage[mathscr]{euscript}
\usepackage[left=2cm, right=2cm, top=2cm, bottom=2cm]{geometry}
\usepackage[usenames, dvipsnames]{color}
\usepackage{multicol}
\usepackage{pgf,tikz}
\usepackage{wrapfig}
\usepackage{mathrsfs}\usetikzlibrary{arrows}
\title{Abbotsleigh 1999 MX2 Trial Q4(b) & solutions}
\author{}
\date{}
\begin{document}
\section*{Abbotsleigh 1999 MX2 Trial Q4(b)}
\textit{Q4 (b)(i)}\\[1em]
Show that the tangent to the rectangular hyperbola $xy = c^2$ at the point $T(ct, \frac{c}{t})$ has equation $x^2 + t^2 y = 2ct$\\[1em]
\textit{Q4 (b)(ii)}\\[1em]
The tangents to the rectangular hyperbola $xy=c^2$ at the points $P(cp,\frac{c}{p})$ and $Q(cq,\frac{c}{q})$, where $pq = 1$, intersect at \(R\).\\[1em]
Find the equation of the locus of $R$ and state any restrictions on the value of $x$ for this locus.
\subsubsection*{Worked Solution:}
\textit{Q4 (b)(i)}\\[1em]
Firstly to find the derivative of the rectangular hyperbola (as a graph)
\begin{flalign*}
&xy=c^2 \Rightarrow y= \frac{c^2}{x} & \hfill \\
&\Rightarrow \frac{dy}{dx} = -\frac{c^2}{x^2}
\end{flalign*}\\
Therefore at $T\left(c, \displaystyle\frac{c}{t}\right)$, $\displaystyle\frac{dy}{dx}=-\displaystyle\frac{c^2}{(ct)^2}$\\
$= -\displaystyle\frac{1}{t^2}$\\[1em]
Now, using the gradient-point form of a line:
\begin{flalign*}
&y-\frac{c}{t} = -\frac{1}{t^2}(x-ct) & \hfill \\
& {t^2}y-ct = -x+ct \\
\end{flalign*}
Therefore the tangent has equation
$$x+t^2y=2ct$$
\newpage\noindent
\textit{Q4 (b)(ii)}\\[1em]
Similar the equations of tangents at $P$ and $Q$ are:
\begin{equation}
x+p^2y=2cp
\end{equation}
and
\begin{equation}
x+q^2y=2cq
\end{equation}
Multiple equation (1) by $q$ and (2) by $p$ to give:
\begin{equation*}
qx+p^2qy=2cpq
\end{equation*}
\begin{equation*}
px+pq^2y=2cpq
\end{equation*}
and since it is given that $pq=1$ , one has equations (3) and (4) respectively:
\begin{equation}
qx=2c-py
\end{equation}
\begin{equation}
px=2c-qy
\end{equation}
Intersect equations (3) and (4) to find $R$, that is (4) $-$ (3):
$$(p-q)x =(p-q)y$$
Divide through by $p-q$ to give the locus of $R$ as the line $y=x$.\\[1em]
But $x=0$ given $c\ne 0$, when substituted into (1) and (2) imply that $p=0$ and $q=0$\\, which is a contradiction.
Therefore, the locus of $R$ is the line $y=x$ where $x \ne 0$

\end{document}
