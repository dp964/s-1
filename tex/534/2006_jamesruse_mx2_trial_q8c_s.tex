\documentclass[11pt,a4paper]{article}
\usepackage[utf8]{inputenc}
\usepackage{amsmath,amsthm}
\usepackage{amssymb}
\usepackage[english]{babel}
\usepackage{changepage} 
\usepackage[mathscr]{euscript}
\usepackage[left=2cm, right=2cm, top=2cm, bottom=2cm]{geometry}
\usepackage[usenames, dvipsnames]{color}
\usepackage{multicol}
\usepackage{pgf,tikz}
\usepackage{wrapfig}
\usepackage{mathrsfs}\usetikzlibrary{arrows}
    \title{James Ruse 2006 MX2 Trial Q8(c) & solutions}
    \author{}
    \date{}
    \begin{document}
\section*{James Ruse 2006 MX2 Trial Q8(c)}
Given $a_1, a_2, a_3, \dots, a_n$ and $b_1, b_2, b_3, \dots, b_n$ are positive real numbers,\\
where $A_n = a_1 + a_2 + a_3 + \cdots + a_n$ and $B_n = b_1 + b_2 + b_3 + \cdots + b_n$ are such that\\
$a_1, a_2, a_3, \dots, a_n > 0$, $b_1, b_2, b_3, \dots, b_n > 0$ and $A_r \leq B_r$, for $r=1,2,3,\dots, n$\\[1em]
\textit{Q8 (c)(i)}\\[1em]
Prove by mathematical induction for $n=1,2,3, \dots$ that:
\begin{align*}
\frac{1}{\sqrt{b_n}}\,B_n+\left(\frac{1}{\sqrt{b_{n-1}}}-\frac{1}{\sqrt{b_n}}\right)B_{n-1}+\left(\frac{1}{\sqrt{b_{n-2}}}-\frac{1}{\sqrt{b_{n-1}}}\right)B_{n-2} + \cdots + \left(\frac{1}{\sqrt{b_1}}-\frac{1}{\sqrt{b_2}}\right)B_{1}\\ = \sqrt{b_1} + \sqrt{b_2} + \sqrt{b_3} + \cdots + \sqrt{b_n} \end{align*}\\
\textit{Q8 (c)(ii)}\\[1em]
Hence, given:
\begin{align*}
\frac{a_1}{\sqrt{b_1}}+\frac{a_2}{\sqrt{b_2}}+\frac{a_3}{\sqrt{b_3}}+\cdots+\frac{a_n}{\sqrt{b_n}}\\
=\frac{1}{\sqrt{b_n}}\,A_n+\left(\frac{1}{\sqrt{b_{n-1}}}-\frac{1}{\sqrt{b_n}}\right)A_{n-1}+\left(\frac{1}{\sqrt{b_{n-2}}}-\frac{1}{\sqrt{b_{n-1}}}\right)A_{n-2}
+ \cdots + \left(\frac{1}{\sqrt{b_{1}}}-\frac{1}{\sqrt{b_2}}\right)A_{1} \end{align*}
\textit{Q8 (c)(iii)}\\[1em]
Deduce that: $\displaystyle\sum_{r=1}^{n}{\sqrt{a_r}} \leq \displaystyle\sum_{r=1}^{n}{\sqrt{b_r}}$
\subsubsection*{Worked Solutions:}
\textit{Q8 (c)(i)}\\[1em]
\underline{For $n=1$:} This case is trivial.\\[1em]
\underline{For $n=2$:}
\begin{flalign*}
\text{LHS} &=\frac{1}{\sqrt{b_2}}\,B_2 + \left(\frac{1}{\sqrt{b_1}}-\frac{1}{\sqrt{b_2}}\right)B_1 &\hfill \\
&=\frac{1}{\sqrt{b_2}}(b_1+b_2) + \left(\frac{1}{\sqrt{b_1}}-\frac{1}{\sqrt{b_2}}\right)b_1\\
&=\frac{b_1}{\sqrt{b_2}}+\sqrt{b_2} + \sqrt{b_1}-\frac{b_1}{\sqrt{b_2}}\\
&=\sqrt{b_2} \text{\quad as required. Therefore, true for $n=2$}
\end{flalign*}\\
Assume result holds up to some $n=k$ (strong induction), that is
$$\frac{1}{\sqrt{b_k}}\,B_k+\left(\frac{1}{\sqrt{b_{k-1}}}-\frac{1}{\sqrt{b_k}}\right)B_{k-1}
+ \cdots + \left(\frac{1}{\sqrt{b_{1}}}-\frac{1}{\sqrt{b_2}}\right)B_{1}=\sum_{r=1}^{k}{\sqrt{b_r}}$$\\
\underline{For $n=k+1$}
\begin{flalign*}
\text{LHS} &=\frac{1}{\sqrt{b_{k+1}}}(B_{k+1}+\left(\frac{1}{\sqrt{b_{k}}}-\frac{1}{\sqrt{b_{k+1}}}\right)B_{k}+\left(\frac{1}{\sqrt{b_{k-1}}}-\frac{1}{\sqrt{b_k}}\right)B_{k-1}
+ \cdots + \left(\frac{1}{\sqrt{b_{1}}}-\frac{1}{\sqrt{b_2}}\right)B_{1} &\hfill \\
&= \frac{1}{\sqrt{b_{k+1}}}(B_{k+1}-B_k)+\left(\frac{1}{\sqrt{b_k}}\,B_k
+\left(\frac{1}{\sqrt{b_{k-1}}}-\frac{1}{\sqrt{b_k}}\right)B_{k-1}
+ \cdots + \left(\frac{1}{\sqrt{b_{1}}}-\frac{1}{\sqrt{b_2}}\right)B_{1}\right)\\
&= \frac{1}{\sqrt{b_{k+1}}}(b_{k+1}) +\sum_{r=1}^{k}{\sqrt{b_r}}\\
&= \sqrt{b_{k+1}} +\sum_{r=1}^{k}{\sqrt{b_r}}\\
&= \sum_{r=1}^{k+1}{\sqrt{b_r}}
\end{flalign*}\\
Therefore, since the initial case and two consecutive cases hold, by the principle of Mathematical Induction, the proposition is true $\qed$.\\[1em]
\textit{Q8 (c)(ii)}\\
\begin{flalign*}
\sum_{r=1}^{n}{\frac{a_r}{\sqrt{b_r}}} &=
\frac{1}{\sqrt{b_k}}\,A_n+\left(\frac{1}{\sqrt{b_{n-1}}}-\frac{1}{\sqrt{b_n}}\right)A_{n-1}
+ \cdots + \left(\frac{1}{\sqrt{b_{1}}}-\frac{1}{\sqrt{b_2}}\right)A_{1} & \hfill \\
&\leq \frac{1}{\sqrt{b_k}}\,B_n+\left(\frac{1}{\sqrt{b_{n-1}}}-\frac{1}{\sqrt{b_n}}\right)B_{n-1}
+ \cdots + \left(\frac{1}{\sqrt{b_{1}}}-\frac{1}{\sqrt{b_2}}\right)B_{1} \quad\text{since $A_i \leq B_i$ for all $0<i\leq n$}\\
& = \sum_{r=1}^{n}{\sqrt{b_r}}
\end{flalign*}
\newpage\noindent
\textit{Q8 (c)(iii)}\\[1em]
Since all $a_r>0$ and $b_r>0$ for all $0<r\leq n$, therefore from Q8(b)(ii)
$$\sqrt{a_r}<\frac{1}{2}\left(\frac{a_r}{\sqrt{b_r}}+\sqrt{b_r} \right),\quad\text{for all $r$ where $0<r\leq n$}$$\\
Summing over $n$ terms gives:
\begin{flalign*}
\sum_{r=1}^{n}{\sqrt{a_r}} &\leq \sum_{r=1}^{n}{\frac{1}{2}\left(\frac{a_r}{\sqrt{b_r}}+\sqrt{b_r} \right)}  &\hfill\\
&= \frac{1}{2} \left(\sum_{r=1}^{n}{\frac{a_r}{\sqrt{b_r}}}\right)+\frac{1}{2}\left(\sum_{r=1}^{n}{\sqrt{b_r}}\right) \quad \text{(rearranging the sum)}\\
&\leq  \frac{1}{2} \left(\sum_{r=1}^{n}{\sqrt{b_r}}\right)+\frac{1}{2}\left(\sum_{r=1}^{n}{\sqrt{b_r}}\right) \quad \text{(using the result from Q8(c)(ii))}\\
&= \sum_{r=1}^{n}{\sqrt{b_r}} \quad \text{as required}\\
\end{flalign*}
\end{document}