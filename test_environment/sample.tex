% <html><body>
\documentclass[11pt]{article}
    \usepackage[utf8]{inputenc}
    \usepackage{amsmath,amsthm}
    \usepackage{amsthm}
    \usepackage{amssymb}
\usepackage{setspace}
    \usepackage[english]{babel}
    \usepackage{changepage} 
    \usepackage[mathscr]{euscript}
    \usepackage[margin=0.8in]{geometry}
    \usepackage[usenames, dvipsnames]{color}
    \usepackage{enumitem}
\usepackage{soulutf8}
\usepackage{titlesec}
\newtheoremstyle{thm}{\topsep}{\topsep}%
     {}%         Body font
     {}%         Indent amount (empty = no indent, \parindent = para indent)
     {}% Thm head font
     {}%        Punctuation after thm head
     {\newline}
     {\underline{\thmname{#1}\thmnumber{ #2}:\thmnote{ #3}}\vpi}

\theoremstyle{thm}

\newtheorem{theorem}{Theorem}
\newtheorem{example}{Example}

\newtheorem{corollary}[theorem]{Corollary}
\newtheorem{lemma}[theorem]{Lemma}

\titleformat{\subsection}[hang]
{\normalsize\bfseries}{\thesubsection}{0.5em}{}[{\titlerule[0.1pt]}]
\renewcommand\thesubsection{\arabic{subsection}.}
\titleformat{\subsubsection}[hang]{\normalsize\itshape\color{mp}}{}{0.5em}{}

\titlespacing*{\subsection}{0px}{1em}{1.5em}
\titlespacing*{\subsubsection}{0px}{1em}{0.5em}
    \definecolor{mp}{RGB}{0, 0, 255}

\newcommand{\textem}[1]{\textcolor{mp}{\textbf{#1}}}
\newcommand{\bluetxt}[1]{\textcolor{mp}{#1}}
\newcommand{\dbhr}[0]{\hrule width \hsize \kern 0.6mm \hrule width \hsize \vspace{0.1cm}}
\newcommand{\vpi}[0]{\medskip \par \noindent}
\renewenvironment{proof}[1][\proofname]{{\noindent \itshape #1:\vpi}}{$\qed$\vpi}

\makeatletter
\newsavebox\myboxA
\newsavebox\myboxB
\newlength\mylenA

\newcommand*\xoverline[2][0.75]{%
    \sbox{\myboxA}{$\m@th#2$}%
    \setbox\myboxB\null% Phantom box
    \ht\myboxB=\ht\myboxA%
    \dp\myboxB=\dp\myboxA%
    \wd\myboxB=#1\wd\myboxA% Scale phantom
    \sbox\myboxB{$\m@th\overline{\copy\myboxB}$}%  Overlined phantom
    \setlength\mylenA{\the\wd\myboxA}%   calc width diff
    \addtolength\mylenA{-\the\wd\myboxB}%
    \ifdim\wd\myboxB<\wd\myboxA%
       \rlap{\hskip 0.5\mylenA\usebox\myboxB}{\usebox\myboxA}%
    \else
        \hskip -0.5\mylenA\rlap{\usebox\myboxA}{\hskip 0.5\mylenA\usebox\myboxB}%
    \fi}
\makeatother

    \title{MATH372 Examples	 25/08/17}
    \author{}
    \date{}
    \begin{document}
\maketitle
\subsection*{Examples of Calculating Metric \& LC Connection}
\begin{example}[$\mathbb{S}^1$ (standard)]
$\mathbb{S}^1$ is the image of the map: $\gamma: [0,2\pi)\rightarrow \mathbb{R}^2,$\\
$\gamma(\theta)=(\cos{\theta},\sin{\theta})	$\\[1em]
The induced metric is $g_{ij}=\Big\langle \partial_i\gamma, \partial_j\gamma\Big\rangle_{\mathbb{R}^2}\,(\theta)$\\[1em]
That is $g_{1,1}=\Big\langle \partial_1\gamma, \partial_1\gamma\Big\rangle_{\mathbb{R}^2}\,(\theta)=|\gamma'|\,(\theta) = |(-\sin{\theta}, \cos{\theta})|^2=1$\\[1em]
Arg length is $|\partial_\theta \gamma|=1$ so the arc length derivatives $\displaystyle\frac{\partial}{\partial s}=\displaystyle\frac{\partial}{\partial \theta}$\\
\end{example}
\begin{example}[Other Circles]
$\partial_r (\theta)=r(\cos{\theta}, \sin{\theta})$, $r>0, r\in \mathbb{R},$ $\theta$ as beforefore in (1)\\[1em]
Now $g_{1,1}(\theta)=\Big|\partial_\theta \gamma_r\Big|^2_{\mathbb{R}^2}\,(\theta) = \Big|r(-\sin{\theta}, \cos{\theta})\Big|^2_{\mathbb{R}^2}=r^2$\\[1em]
What is the metric? It gives the product $X,Y \in T_0\,\mathbb{S}^1$: $\big\langle X, Y \big\rangle_g = g_{ij}X^i Y^j =r^2 X\,Y$\\[1em]
Here one has:
$\displaystyle\frac{\partial}{\partial s}= \displaystyle\frac{1}{|\partial_\theta \gamma|}\displaystyle\frac{\partial}{\partial \theta}=\sqrt{g_{11}}\,\displaystyle\frac{\partial}{\partial \theta}=\displaystyle\frac{1}{r}\displaystyle\frac{\partial}{\partial \theta}$\\
\vpi
\textit{Christoffel Symbols}\\
$\Gamma_{ij}^k = \displaystyle\frac{1}{2}g^{kl}(\partial_i g_{il}+\partial_j g_{ij} - \partial_l g_ij) = 0, g$ is constant\\[1em]
Note: $\displaystyle\frac{\partial}{\partial s}=\displaystyle\frac{\partial}{\partial \theta}$ if and only if $|\partial_\theta \gamma|=1$\\
\end{example}

\begin{example}[The Spiral]
	$\partial(\theta)=e^{\theta}(\cos{\theta}, \sin{\theta})$\\[1em]
	$g_{11}(\theta)=|\partial_\theta \gamma|^2 =|e^\theta (\cos{\theta}, \sin{\theta})+e^\theta (-\sin{\theta}, \cos{\theta})|^2 = 2e^{2\theta}$\\[0.5em]
	$\Rightarrow \displaystyle\frac{\partial}{\partial s}=\displaystyle\frac{1}{\sqrt{2} e^\theta}\displaystyle\frac{\partial}{\partial \theta}$\\
\end{example}
\newpage
\begin{example}[$\mathbb{S}^2$ (standard)]
For $f: \mathscr{D} \rightarrow \mathbb{S}^2, f(x,y)=(x,y,\sqrt{1-x^2-y^2})$\\[1em]
The metric is $g_{ij}=\big\langle \partial_i f, \partial_j f \big\rangle_{\mathbb{R}^3}$, and so\\
$g_{11}=\big\langle \partial_x f, \partial_x f \big\rangle_{\mathbb{R}^3} = \left\langle \left(1, 0, \displaystyle\frac{-x}{\sqrt{1-x^2-y^2}} \right) , \left(1, 0, \displaystyle\frac{-x}{\sqrt{1-x^2-y^2}} \right) \right\rangle_{\mathbb{R}^3} $\\
$= \displaystyle\frac{1-y^2}{1-x^2-y^2}$\\[0.5em]
$g_{22} = \displaystyle\frac{1-x^2}{1-x^2-y^2}$\\[0.5em] $g_{12}=g_{21}=\displaystyle\frac{xy}{1-x^2-y^2}$\\[1em]
$g_{ij}=\displaystyle\frac{1}{1-x^2-y^2} \begin{pmatrix}
	1-y^2       & xy \\
	xy       & 1-x^2
\end{pmatrix}$\\[1em]
Aside: $d \mu = \sqrt{\det{g}}$ - induced measure.\\
\vpi
Now for Riemannian matrices this matrix has the property of being positive definite.\\[1em]
$\det{g}=\displaystyle\frac{(1-y^2)(1-x^2)-x^2 y^2}{(1-x^2-y^2)^2}=\displaystyle\frac{1-x^2-y^2}{(1-x^2-y^2)^2}=\displaystyle\frac{1}{(1-x^2-y^2)}$\\
\vpi
\textit{Christoffel Symbols}\\
$\Gamma_{ij}^k = \displaystyle\frac{1}{2}g^{kl}(\partial_i g_{il}+\partial_j g_{ij} - \partial_l g_ij)$, $i,j,k = 1,2.$\\[1em]
Now $\Gamma_{ij}^{k}=\Gamma_{ji}^{k}$, so there are $\Gamma_{11}^{1}, \Gamma_{11}^{2}, \Gamma_{22}^{2}, \Gamma_{22}^{2}, \Gamma_{12}^{2}, \Gamma_{12}^{2}$ and
$\nabla_i X_j = \partial_i X_j + \Gamma_{ij}{k}X_k$\\
(and so on, the calculation is long)
\end{example}








\end{document}
% </body></html>