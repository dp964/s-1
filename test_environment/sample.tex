\documentclass[11pt]{report}
    \usepackage[utf8]{inputenc}
    \usepackage{amsmath,amsthm}
    \usepackage{amsthm}
    \usepackage{amssymb}
\usepackage{setspace}
    \usepackage[english]{babel}
    \usepackage{changepage} 
    \usepackage[mathscr]{euscript}
    \usepackage[margin=0.8in]{geometry}
    \usepackage[usenames, dvipsnames]{color}
    \usepackage{enumitem}
\usepackage{soulutf8}
\usepackage{titlesec}
\newtheoremstyle{thm}{\topsep}{\topsep}%
     {}%         Body font
     {}%         Indent amount (empty = no indent, \parindent = para indent)
     {}% Thm head font
     {}%        Punctuation after thm head
     {\newline}
     {\underline{\thmname{#1}\thmnumber{ #2}:\thmnote{ #3}}\vpi}

\theoremstyle{thm}

\newtheorem{theorem}{Theorem}
\newtheorem{example}{Example}

\newtheorem{corollary}[theorem]{Corollary}
\newtheorem{lemma}[theorem]{Lemma}


\titleformat{\subsection}[hang]
{\normalsize\bfseries}{\thesubsection}{0.5em}{}[{\titlerule[0.1pt]}]
\renewcommand\thesubsection{\arabic{subsection}.}
\titleformat{\subsubsection}[hang]{\normalsize\itshape\color{mp}}{}{0.5em}{}

\titlespacing*{\subsection}{0px}{1em}{1.5em}
\titlespacing*{\subsubsection}{0px}{1em}{0.5em}
    \definecolor{mp}{RGB}{0, 0, 255}

\newcommand{\textem}[1]{\textcolor{mp}{\textbf{#1}}}
\newcommand{\bluetxt}[1]{\textcolor{mp}{#1}}
\newcommand{\dbhr}[0]{\hrule width \hsize \kern 0.6mm \hrule width \hsize \vspace{0.1cm}}
\newcommand{\vpi}[0]{\medskip \par \noindent}
\renewenvironment{proof}[1][\proofname]{{\noindent \itshape #1:\vpi}}{$\qed$\vpi}

\makeatletter
\newsavebox\myboxA
\newsavebox\myboxB
\newlength\mylenA

\newcommand*\xoverline[2][0.75]{%
    \sbox{\myboxA}{$\m@th#2$}%
    \setbox\myboxB\null% Phantom box
    \ht\myboxB=\ht\myboxA%
    \dp\myboxB=\dp\myboxA%
    \wd\myboxB=#1\wd\myboxA% Scale phantom
    \sbox\myboxB{$\m@th\overline{\copy\myboxB}$}%  Overlined phantom
    \setlength\mylenA{\the\wd\myboxA}%   calc width diff
    \addtolength\mylenA{-\the\wd\myboxB}%
    \ifdim\wd\myboxB<\wd\myboxA%
       \rlap{\hskip 0.5\mylenA\usebox\myboxB}{\usebox\myboxA}%
    \else
        \hskip -0.5\mylenA\rlap{\usebox\myboxA}{\hskip 0.5\mylenA\usebox\myboxB}%
    \fi}
\makeatother

    \title{MATH372 Notes}
    \author{Daniel P}
    \date{}
    \begin{document}
    	    \maketitle
\part{Differential Geometry}
\chapter{What are Manifolds?}
\section{Definition of a Manifold}
A \textem{$C^k$-manifold} $\mathcal	{M}^n$ is a locally Euclidean "nice" 
topological space.
\subsubsection{1. Manifolds as topological spaces}
A \textem{topological space}, with topology $\mathcal{T}$ 
(where $\mathcal{M}^n\in \mathcal{T}$) is such that
	\begin{itemize}
\item A set $\mathcal{U}\in \mathcal{T}$ is called \textit{open}
\item Any finite or infinite union of open sets is open and in $\mathcal{T}$ 
that is for $\{\mathcal{U}_1, \cdots,\,\mathcal{U}_k\}$, the set 
$\bigcup_{i=1}^{k}{\mathcal{U}_i}\in \mathcal{T}$ or for 
$\{\mathcal{U}_1,\cdots,  \,\mathcal{U}_k,\cdots \}$, the set 
$\bigcup_{i=1}^{\infty}{\mathcal{U}_i}\in \mathcal{T}$
		\item Any finite intersection of open sets is open and in $\mathcal{T}$ 
		that is for $\{\mathcal{U}_1, ...,  \mathcal{U}_k\}$, the set 
		$\bigcap_{i=1}^{k}{\mathcal{U}_i}\in \mathcal{T}$
\end{itemize}
\subsubsection{2. Hausdorff and Second-Countable}
However, the topology on manifold needs to be "nice" and so the following 
restrictions are applied:
\begin{itemize} 
\item 
The topology is \textem{Hausdorff}, meaning for $p\ne q$, then there exists open sets $\mathcal{U}$ and an open set $\mathcal{V}$ such that $p\in \mathcal{U}, q\in \mathcal{V}$ and $\mathcal{U} \cap \mathcal{V} = \emptyset$
\item The topology is \textem{second-countable}, meaning it has a countable 
base $\mathcal{F}$ where any $\mathcal{U}\in \mathcal{T}$ can be written as the 
following countable union 
$\mathcal{U}=\displaystyle\bigcup_{i=1}^{\infty}(\mathcal{U}_i),$ given that 
$\mathcal{U}_i\in \mathcal{F}$ or equivalently for all $x\in 
\mathcal{M}^n,\;\mathcal{U}_\alpha \in \mathcal{T},\; 	x\in 
\mathcal{U}_\alpha,\;$ there exists a $\mathcal{U}\in F$ such that $x\in 
\mathcal{U} \subset \mathcal{U}_\alpha$, where $F$ is countable.
\end{itemize}
\subsubsection{3. Locally Euclidean}
Also as per the definition, the manifold is \textem{locally Euclidean}.\\[1em] 
This means that there exists an open 
cover $\{\mathcal{U}_i\}_{i \in \mathcal{A}} \subset \mathcal{T}$ and injective 
(one-to-one) maps: $x_i: \mathcal{U}_i \rightarrow \mathbb{R}^n$ such 
that for any two \textit{charts} $(\mathcal{U}_1, x_1),\; 
(\mathcal{U}_2, x_2)$ with $U_1 \cap\, U_2 \ne \emptyset$ (non-empty domain) 
the \textit{transistion map} $x_1 \circ x_2^{-1} : 
x_2(\mathcal{U}_1 \cap\, \mathcal{U}_2) \rightarrow 
x_1(\mathcal{U}_1 \cap\, \mathcal{U}_2)$ is a 
\textit{$C^k$-homeomorphism} (open sets mapped into open sets)\\[1em]
The \textem{atlas} is the set containing all the charts 
$\mathcal{A}=\{x_i: \mathcal{U}_i \rightarrow \mathbb{R}^n\}$, where 
$x_\alpha$ is a chart.
\textit{Note:} Smooth (analytic) functions in $\mathbb{R}^n$ are homeomorphisms.
\newpage\noindent \subsubsection{4. Notes on Definition of Manifold}
\begin{itemize}
	\item The transition map $x_\alpha \circ x_\beta^{-1}$ is a map between 
	Euclidean spaces so differentiation and integration makes sense on this 
	function (using methods learnt in multivariate \& vector caculus)
	\item For a set to be closed, given a trivial topological space 
	$(\mathcal{M}^n, \mathcal{T})$, consider the open set $\mathcal{U}$. Its 
	compliment $\mathcal{M}^n \setminus \mathcal{U}$ is closed.
\item Two trivial examples of topological spaces are $\mathcal{M}^n = 
\emptyset, \mathcal{T} = \{\emptyset\}$ and $\mathcal{M}^n = X, \mathcal{T}=\{X,
\emptyset\}$\\
\end{itemize}
\section{Examples of Manifolds: Euclidean Space}
Putting aside the trivial cases, claim that Euclidean space itself, that is 
$\mathbb{R}^n$ is a smooth manifold.\\[1em]
To verify the claim requires finding and defining the topology $\mathcal{T}$ as 
well as an atlas $\mathcal{A}$ containing an appropriate set of charts.
\subsubsection{1. Set the Topology}
Firstly $\mathbb{R}^n$ needs to be defined as a topological space. Define 
$\mathbb{R}^n$ as the set $\mathbb{R}^n: \{(x_1, \cdots, x_n): x_i \in 
\mathbb{R}\}$.\\[1em]
Consider the \textit{distance} function $d: \mathbb{R}^n \times \mathbb{R}^n 
\rightarrow [0,\infty)$ given by 
$d(x,y)=|x-y|=\sqrt{{(x_1-y_1)}^2+\cdots+{(x_n-y_n)}^2}$ and define the 
\textit{open ball} $B_r(x)=\{y\in \mathbb{R}^n: d(x,y)<r\}$ for all $r>0$ and 
$x\in \mathbb{R}^n$.\\[1em]
To construct the topology $\mathcal{T}$, it is done iteratively:
\begin{enumerate}
	\item Let $\emptyset \in \mathcal{T}$ and $\mathbb{R}^n \in \mathcal{T}$
	\item Let ${B_r(x)}\in \mathcal{T}$
	\item Let ${\mathcal{T}_0 =\{B_r(x): x\in \mathbb{R}^n,\; r > 0\}}$. Then the set containing all the unions $\left\{\bigcup\limits_{S\in T}{S} : T\subset \mathcal{T}_0 \right\} \in \mathcal{T}$
	\item Let $\{r_1, \cdots, r_m \},\; \{x_1, \cdots, x_m\}$ be finite sequences, where $r_i \in (0,\infty), x_i \in \mathbb{R}^n$.\\[0.5em]
	Then  $\bigcap\limits_{i=1}^{m}{B_{r_i}(x_i)} \in \mathcal{T}$
\end{enumerate}
The topology $\mathcal{T}^{(\mathbb{R}^n)}$ is said to be generated by the 
distance (function).
\subsubsection{2. Check the topology is Hausdorff:}
Let $p,q\in \mathbb{R}^n$, where $p\ne q$, that is $p,q$ are distinct points 
within $\mathbb{R}^n$\\[1em]
If $p\ne q$ then $|p-q|>0$.\\[1em]
Choose $\mathcal{U}_p = B_{\frac{|p-q|}{4}}(p)\in \mathcal{T}$ and $\mathcal{U}_q = B_{\frac{|p-q|}{4}}(q)\in \mathcal{T}$\\[1em]
The choice of radius was sufficient to ensure that $\mathcal{U}_p \cap 
\mathcal{U}_q = \emptyset$.\\[1em]
Since $p,q$ were arbitrary, the topology is Hausdorff.$\qed$
\newpage\noindent
\subsubsection{3. Check the topology is Second Countable:}
It is known that $\mathbb{Q}$ and $\mathbb{Q}^n$ are countable and therefore so 
is the set $\mathcal{F}=\{B_r(x): x \in \mathbb{Q}^n,\; r\in 
(0,\infty)\,\cap\,\mathbb{Q} \}$. This set is the set of open balls with 
rational centres and positive rational radii.\\[1em]
Claim that $F\subset \mathcal{T}$ is the countable base required.\\[1em]
To do so the following observations are made:
\begin{itemize}
	\item Every real number $r$ can be written as a Cauchy sequence of $r_i\in \mathbb{Q}$ such that $r_i \nearrow r$
	\item Similarly, every $x$ can be written as a Cauchy sequences of $x_i\in \mathbb{Q}^n $ such that $x_i \nearrow x$
	\item Then consider the ball $B_r(x)$ where $r\in (0,\infty), x\in \mathbb{R}^n$ can be written as $B_r(x)=\bigcup\limits_{i=1}^{\infty}{\mathcal{U}_i}$ where $\mathcal{U}_i \in \mathcal{F}$
\end{itemize}
However, there is a problem, sets in $\mathcal{T}$ can be uncountable unions, 
meaning it is unknown whether the union, as above, is countable or not.\\[1em] 
To resolve this and to avoid an uncountable union, requires the fact that 
$\mathbb{Q}^n$ is dense in $\mathbb{R}^n$.\\[1em]
Pick an arbitrary $p\in \mathbb{R}^n$ and an open set$ \mathcal{U}_0 \in 
\mathcal{T}$ such that $p\in \mathcal{U}_0$.\\[1em]
The Hausdorff property of $\mathbb{R}^n$ means that there exists an $\varepsilon$ such that $B_\varepsilon(p) \subset \mathcal{U}_0$.\\[1em]
The density of $\mathbb{Q}$ means that one can take a $\delta$ such that $\delta \in  (\frac{\varepsilon}{2},\varepsilon) \cap \mathbb{Q}$\\[1em]
Since a rational centre is needed also, take a $q \in \mathbb{Q}^n$ such that $d(p,q)<\displaystyle\frac{\varepsilon}{8}$ (using the density of $\mathbb{Q}^n$).
Then $B_\delta(q) \in F$ and $p\in B_\delta(q)\subset \mathcal{U}_0$, and since 
the choice of $p$ was arbitrary, this completes the proof.
\subsubsection{4. Construction of the Atlas}
Recall that an atlas is a set $\mathcal{A} = \left\{ x_\alpha: 
\mathcal{U}_\alpha \rightarrow \mathbb{R}^n \text{ where } \alpha \in 
A,\mathcal{U}_\alpha \in \mathcal{T},  \bigcup\limits_{\alpha \in 
A}{\mathcal{U}_\alpha}=\mathcal{M}^n,\text{ and } x_\alpha\text{ is injective} 
\right\}$ and for each $x_{\alpha}, x_{\beta} \in A$, the map $x_\alpha \circ 
x_\beta^{-1} : x_\beta(\mathcal{U}_\alpha \cap \mathcal{U}_\beta) \rightarrow 
x_\alpha(\mathcal{U}_\alpha \cap \mathcal{U}_\beta)$ is a smooth 
homeomorphism\\[1.5em]
Let $x_0: \mathbb{R}^n \rightarrow \mathbb{R}^n$ be given by $x_0(x)=x$, that is the identity map, and claim that $A=\{x_0\}$ is an atlas.\\[1em]
Claim that $x_0$ is a chart. Its domain is $\mathbb{R}^n \in \mathcal{T}$ and 
the target is $\mathbb{R}^n$ and $\mathbb{R}^n$ covers $\mathbb{R}^n$.\\[1em]
It is also injective as $x_0(x)=x_0(y)\Rightarrow x=y$ (it is therefore a 
bijection)\\[1em]
Claim that the transistion maps are homeomorphisms. Let $\tau = x_0 \circ 
x_0^{-1}$. Then $\tau: \mathbb{R}^n \rightarrow \mathbb{R}^n$ is simply the map 
$\tau(x)$ which is the identity map and is a smooth (analytic) homeomorphism.
\subsubsection{5. Conclusion}
It is possible to conclude that $(\mathbb{R}^n, \mathcal{T}, \mathcal{A})$ is a smooth manifold where $\mathcal{T}$ is the topology generated by the distance function (or the notion of the 'open ball') and $\mathcal{A}=\{x \mapsto x \}$\\[1em]
\textit{Discussion:} Are there other charts that could be used to form the atlas for $\mathbb{R}^n$?
\newpage\noindent
\begin{example}[$\mathbb{S}^1$ as a manifold]
The manifold $\mathbb{S}$ is simply the circle, it can be defined $\mathbb{S}^1 = \{x \in \mathbb{R}^2: |x|=1 \}$\\
\vpi
\bluetxt{2.1 Set the Topology:}\\[1em]
Let $\mathcal{T}^{(\mathbb{S}^1)} = \{\mathcal{U} \subset \mathbb{S}^1: \mathcal{U}=\mathcal{U}_0 \cap \mathbb{S}^1,\text{ where } \mathcal{U}_0 \in \mathcal{T}^{(\mathbb{R}^2)} \}$.\\
\textit{Remark:} This is called the \textem{subspace topology}\\
\vpi
\bluetxt{2.2A Check the topology is Hausdorff:}\\[1em]
Claim that the desired property of the topology is inherited from $\mathbb{R}^2$. So let $p, q \in \mathbb{S}^1$ with $p \ne q$.\\[1em]
Since $\mathbb{R}^2$ is Hausdorff, there exists $\mathcal{U}_p,\; \mathcal{U}_q$ such that $\mathcal{U}_p,\;\mathcal{U}_q \in \mathcal{T}^{(\mathbb{R}^2)}$, and $p \in \mathcal{U}_p,\; q \in \mathcal{U}_q,\; \mathcal{U}_p\cap\,\mathcal{U}_q=\emptyset.$\\[1em]
Let $\mathcal{V}_p =\mathcal{U}_p\cap\,\mathbb{S}^1 \in \mathcal{T}^{(\mathbb{S}^1)},\; \mathcal{V}_q =\mathcal{U}_q\cap\,\mathbb{S}^1 \in \mathcal{T}^{(\mathbb{S}^1)}$ and $p\in \mathcal{V}_p, q\in \mathcal{V}_q,$ then $ \mathcal{V}_p \cap\,\mathcal{V}_q = \emptyset$ as required.\\[1em]
\textit{Discussion:} Does "inherited" make sense here?
\newpage \noindent
\bluetxt{2.2B Check the topology is Second Countable:}\\[1em]
Claim that the desired property is inherited from $\mathbb{R}^2$.\\[1em]
Let $\mathcal{F}^{(\mathbb{S}^1)}= \left\{\mathcal{V}: \mathcal{V} = \mathcal{U}\cap\,\mathbb{S}^1 \text{ where } \mathcal{U} \in \mathcal{F}^{(\mathbb{R}^2)} \right\}$, where $\mathcal{F}^{(\mathbb{R}^2)}$ is the countable base in $\mathbb{R}^2$.\\[1em]
Since $\mathcal{F}^{(\mathbb{S}^1)} \subset \mathcal{F}^{(\mathbb{R}^2)}$, it is countable. Now it remains to check that $\mathcal{F}^{(\mathbb{S}^1)}$ is a base.\\[1em]
Let $p\in \mathbb{S}^1,\; \mathcal{V}_0 \in \mathcal{T}$ such that $p\in \mathcal{V}_0$. Since $\mathcal{V}_0 \in \mathcal{T}^{(\mathbb{S}^1)}, \mathcal{V}_0 = \mathcal{U}_0 \cap\, \mathbb{S}^1$ , where $U_0 \in \mathcal{T}^{(\mathbb{R}^2)}$.\\[1em]
Then since $\mathcal{F}^{(\mathbb{R}^2)}$ is a base, there exists a $\mathcal{U}_1 \subset \mathcal{U}_0$ such that $p \in \mathcal{U}_1$ and $ \mathcal{U}_1 \in \mathcal{F}^{(\mathbb{R}^2)} \subset \mathcal{T}^{(\mathbb{R}^2)}$.\\[1em]
Take $\mathcal{V}_1 = \mathcal{U}_1 \cap\, \mathbb{S}^1$. Then $p\in \mathcal{V}_1 \in \mathcal{F}^{(\mathbb{S}^1)}$, where $\mathcal{V}_1 \subset \mathcal{V}_0$ as required.\\
\vpi
\bluetxt{2.3 Construction of the Atlas:}\\[1em]
The idea behind constructing the atlas is to write the manifold as graphs over its tangent planes.\\[1em]
\textit{Atlas \#1: Using 4 charts}\\[1em]
Pick open subsets of $\mathbb{S}^1$ and charts as follows:\\
$\mathcal{U}_1: \{(x,y)\in \mathbb{S}^1: y > 0 \}$ and $ x_1: \mathcal{U}_1 \rightarrow \mathbb{R},\; x_1(x,y)= x$\\
$\mathcal{U}_2: \{(x,y)\in \mathbb{S}^1: y < 0 \}$ and $ x_2: \mathcal{U}_2 \rightarrow \mathbb{R},\; x_2(x,y)= x$\\
$\mathcal{U}_3: \{(x,y)\in \mathbb{S}^1: x > 0 \}$ and $ x_3: \mathcal{U}_3 \rightarrow \mathbb{R},\; x_3(x,y)= y$\\
$\mathcal{U}_4: \{(x,y)\in \mathbb{S}^1: x < 0 \}$ and $ x_4: \mathcal{U}_4 \rightarrow \mathbb{R},\; x_4(x,y)= y$\\[1em]
\textit{Discussion:} Are each of the $\mathcal{U}_i$ are in the topology of $\mathbb{S}^1$?\\[1em]
(1) The $\mathcal{U}_i$ form a cover for $\mathbb{S}_1$, as each $\mathcal{U}_i \in \mathcal{T}^{(\mathbb{S}^1)}$ and $\mathcal{U}_1 \cap\, \mathcal{U}_2 \cap  \, \mathcal{U}_3 \cap \, \mathcal{U}_4 = \mathbb{S}^1$\\[1em]
(2) These $x_i$ charts are injective.\\[1em]
Using the definition of $\mathbb{S}^1$ for the positive and negative portions of the circle, one has $y=\pm\sqrt{1-x^2}$ and $x=\pm\sqrt{1-y^2}$\\[1em]
So $x_1(x,y)=x_1(x_0, y_0) \Rightarrow x_1(x, \sqrt{1-x^2})=x_1(x_0, \sqrt{1-x_0^2}) \Rightarrow x=x_0,y=y_0$ $(\text{as } 1-x^2=1-x_0^2) $
and $x_2(x,y)=x_2(x_0, y_0) \Rightarrow x_1(x, -\sqrt{1-x^2})=x_1(x_0, \sqrt{1-x_0^2}) \Rightarrow x=x_0,y=y_0$ $(\text{as } 1-x^2=1-x_0^2) $
and $x_3(x,y)=x_3(x_0, y_0) \Rightarrow x_1(\sqrt{1-y^2}, y)=x_1(\sqrt{1-y_0^2}, y) \Rightarrow y=y_0,x=x_0$ $(\text{as } 1-y^2=1-y_0^2) $
and $x_4(x,y)=x_4(x_0, y_0) \Rightarrow x_1(\sqrt{1-y^2}, y)=x_1(\sqrt{1-y_0^2}, y) \Rightarrow y=y_0,x=x_0$ $(\text{as } 1-y^2=1-y_0^2) $\\[1em]
Claim that $\mathcal{A} =\{x_1, x_2, x_3, x_4\}$ is an atlas for $(\mathbb{S}^1, \mathcal{T}^{(\mathbb{S}^1)})$\\[1em]
(3) The transition maps are continuous.\\[1em]
So $x_1^{-1}(x)=(x,\sqrt{1-x^2})$,
$x_2^{-1}(x)=(x,-\sqrt{1-x^2})$,
$x_3^{-1}(y)=(\sqrt{1-y^2},y)$, and 
$x_4^{-1}(y)=(-\sqrt{1-y^2},y)$\\[1em]
Then $(x_1 \circ x_3^{-1})(x)=x_1(\sqrt{1-y^2}, y)=\sqrt{1-y^2}$ (smooth for $y>0$).\\[1em]
$(x_2 \circ x_4^{-1})(x)=x_2(-\sqrt{1-y^2}, y)=-\sqrt{1-y^2}$ (smooth for $y<0$), similarly for $(x_3 \circ x_1^{-1})$ and $(x_4 \circ x_2^{-1})$\\[1em]
(Note that $\mathcal{U}_1\cap\,\mathcal{U}_2 =\emptyset$ and $\mathcal{U}_3\cap\,\mathcal{U}_4 =\emptyset$, $(x_1 \circ x_2^{-1})=(x_2 \circ x_1^{-1})=x$ and $(x_3 \circ x_4^{-1})=(x_4 \circ x_3^{-1})=y$ but isn't required)
\newpage\noindent
\textit{Atlas \#2: Using 3 charts}\\[1em]
Pick open subsets of $\mathbb{S}^1$ and charts as follows:\\
$\mathcal{U}_1: \{(x,y)\in \mathbb{S}^1: y > 0 \}$ and $ x_1: \mathcal{U}_1 \rightarrow \mathbb{R},\; x_1(x,y)= x$\\
$\mathcal{U}_2: \{(x,y)\in \mathbb{S}^1: x+y < 0 \}$ and $ x_2: \mathcal{U}_2 \rightarrow \mathbb{R},\; x_2(x,y)= \frac{\sqrt{2}}{2}(y-x)$\\
$\mathcal{U}_3: \{(x,y)\in \mathbb{S}^1: x > 0 \}$ and $ x_3: \mathcal{U}_3 \rightarrow \mathbb{R},\; x_3(x,y)= y$\\[1em]
\end{example}
\newpage\noindent
\section{Properties of Manifolds}
\subsubsection{1. Charts and Compatible Charts}
For $(\mathcal{M}^n, \mathcal{T})$ be a Hausdorff, second-countable, 
topological 
space. Recall that a manifold is locally a Euclidean space.\\[1em]
Recall that a \textem{chart} is a pair $(\mathcal{U}, x_i) $ where $U_i 
\subset \mathcal{M}^n$ and $x_i: \mathcal{U} \rightarrow \mathbb{R}^n$ is a 
homeomorphism onto its image (it is an injective map).\\[1em]
The \textem{transition maps} (also called an overlap map), provided are:
$\mathcal{U}_1 
\cap \mathcal{U}_2 \ne \emptyset$\\
$x_1 \circ x_2^{-1} : 
x_\beta(\mathcal{U}_1 \cap\, \mathcal{U}_2) \rightarrow 
x_\alpha(\mathcal{U}_1 \cap\, \mathcal{U}_2)$ and
$x_2 \circ x_1^{-1} : 
x_\beta(\mathcal{U}_1 \cap\, \mathcal{U}_2) \rightarrow 
x_\alpha(\mathcal{U}_1 \cap\, \mathcal{U}_2)$\\[1em]
If these maps are smooth, then the charts are called \textem{compatible}. As as 
before if such maps exist that cover $\mathcal{M}$, then $\mathcal{M}$ is a 
smooth manifold. (Without loss of generality, this also applies to $C^k$ 
transition maps with $C^k$ inverses)
\subsubsection{2. Maximal Atlas}
Set $(\mathcal{M}^n, \mathcal{T}, \mathcal{A})$ be a $C^k$-manifold. Suppose 
that 
for all charts $(\mathcal{U}_i, x_i)$ where $x_i$ is compatible 
with $\mathcal{A}$ (with $x_i \in \mathcal{A}$), then 
$\mathcal{A}$ is called a \textem{maximal atlas}.\\[1em]
Given any manifold, where the atlas $\mathcal{A}$ is not maximal, add any 
charts compatible with $\mathcal{A}$ until it is maximal.\\[1em]
In other terms an atlas is maximal if there does not exist any atlas 
$\mathcal{A}'$ such that $\mathcal{A} \subseteq \mathcal{A}'$
\subsubsection{3. Differentiable maps between manifolds}
Let $(\mathcal{M}_1^n, \mathcal{T}_1, \mathcal{A}_1)$ and $(\mathcal{M}_2^n, 
\mathcal{T}_2,\mathcal{A}_2)$ be smooth manifolds with maximal 
atlases, with respective charts $(\mathcal{U}, x)$ and 
$(\mathcal{V}, y)$ where $x \in \mathcal{A}_1$ and $y \in 
\mathcal{A}_2$ \\[1em]
Then a map between manifolds $f: \mathcal{M}_1 \rightarrow \mathcal{M}_2$ is 
said to be 
\textem{differentiable} at $p\in \mathcal{M}^1$ if given charts $x: \mathcal{U} 
\rightarrow \mathbb{R}^n$ and $y: \mathcal{V} \rightarrow \mathbb{R}^n$ 
(where $p\in \mathcal{U}$ and $f(\mathcal{U})\subset \mathcal{V}$), 
the map $y \circ f \circ x^{-1}: x(\mathcal{U}) \rightarrow 
y(f(\mathcal{U}))$ is differentiable at $x(p) \in \mathbb{R}^n$\\[1em]
\textit{Note:} More generally, the map $f$ is called smooth if the composition 
as before $y \circ f \circ x^{-1}$ is smooth. If for a homeomorphism $f: 
\mathcal{M}_1 \rightarrow \mathcal{M}_2$ and $f^{-1}$ are smooth, then 
$\mathcal{M}_1$ and $\mathcal{M}_2$ are said to be \textit{diffeomorphic}.
\newpage\noindent 
\subsubsection{4. Tangent Vectors}
Let $(\mathcal{M}^n, \mathcal{T}, \mathcal{A})$ be a smooth manifold with 
maximal atlas, then a map $\alpha: (-\varepsilon, \varepsilon) \rightarrow 
\mathcal{M}^n$ that is differentiable at every $\alpha(t)$ is called a 
\textit{differentiable curve}.\\[1em]
Let $\alpha(0)= p \in \mathcal{M}^n$, then $D_p$ = $\{f: \mathcal{M}^n 
\rightarrow \mathbb{R}^n: f \text{ is differentiable at } p\}$.\\[1em]
The \textem{tangent vector} to $\alpha$ at $t=0$ is an \textit{operator}: 
$\alpha'(0): D_p \rightarrow \mathbb{R}$ given by, for all $f \in D_p$, 
$\alpha'(0) f = \displaystyle\frac{d}{dt}(f \circ \alpha)\bigg|_{t=0}$
\chapter{Connections}
\bluetxt{Definition:}\\[1em]
An \textem{afffine connection} $\nabla$ is a map $\nabla : \Gamma(TM)\times 
\Gamma(T)$
\end{document}