% [html]
% [body]
\documentclass[11pt]{article}
    \usepackage[utf8]{inputenc}
    \usepackage{amsmath,amsthm}
    \usepackage{amsthm}
    \usepackage{amssymb}
\usepackage{setspace}
    \usepackage[english]{babel}
    \usepackage{changepage} 
    \usepackage[mathscr]{euscript}
    \usepackage[margin=0.8in]{geometry}
    \usepackage[usenames, dvipsnames]{color}
    \usepackage{enumitem}
\usepackage{soulutf8}
\usepackage{titlesec}
\newtheoremstyle{thm}{\topsep}{\topsep}%
     {}%         Body font
     {}%         Indent amount (empty = no indent, \parindent = para indent)
     {}% Thm head font
     {}%        Punctuation after thm head
     {\newline}
     {\underline{\thmname{#1}\thmnumber{ #2}:\thmnote{ #3}}\vpi}

\theoremstyle{thm}

\newtheorem{theorem}{Theorem}
\newtheorem{example}{Example}

\newtheorem{corollary}[theorem]{Corollary}
\newtheorem{lemma}[theorem]{Lemma}

\titleformat{\subsection}[hang]
{\normalsize\bfseries}{\thesubsection}{0.5em}{}[{\titlerule[0.1pt]}]
\renewcommand\thesubsection{\arabic{subsection}.}
\titleformat{\subsubsection}[hang]{\normalsize\itshape\color{mp}}{}{0.5em}{}

\titlespacing*{\subsection}{0px}{1em}{1.5em}
\titlespacing*{\subsubsection}{0px}{1em}{0.5em}
    \definecolor{mp}{RGB}{0, 0, 255}

\newcommand{\textem}[1]{\textcolor{mp}{\textbf{#1}}}
\newcommand{\bluetxt}[1]{\textcolor{mp}{#1}}
\newcommand{\dbhr}[0]{\hrule width \hsize \kern 0.6mm \hrule width \hsize \vspace{0.1cm}}
\newcommand{\vpi}[0]{\medskip \par \noindent}
\renewenvironment{proof}[1][\proofname]{{\noindent \itshape #1:\vpi}}{$\qed$\vpi}

\makeatletter
\newsavebox\myboxA
\newsavebox\myboxB
\newlength\mylenA

\newcommand*\xoverline[2][0.75]{%
    \sbox{\myboxA}{$\m@th#2$}%
    \setbox\myboxB\null% Phantom box
    \ht\myboxB=\ht\myboxA%
    \dp\myboxB=\dp\myboxA%
    \wd\myboxB=#1\wd\myboxA% Scale phantom
    \sbox\myboxB{$\m@th\overline{\copy\myboxB}$}%  Overlined phantom
    \setlength\mylenA{\the\wd\myboxA}%   calc width diff
    \addtolength\mylenA{-\the\wd\myboxB}%
    \ifdim\wd\myboxB<\wd\myboxA%
       \rlap{\hskip 0.5\mylenA\usebox\myboxB}{\usebox\myboxA}%
    \else
        \hskip -0.5\mylenA\rlap{\usebox\myboxA}{\hskip 0.5\mylenA\usebox\myboxB}%
    \fi}
\makeatother

    \title{MATH372 Examples	 25/08/17}
    \author{}
    \date{}
    \begin{document}
\maketitle
\subsubsection{3. Check the topology is Second Countable:}
It is known that $\mathbb{Q}$ and $\mathbb{Q}^n$ are countable and therefore so 
is the set $\mathcal{F}=\{B_r(x): x \in \mathbb{Q}^n,\; r\in 
(0,\infty)\,\cap\,\mathbb{Q} \}$. This set is the set of open balls with 
rational centres and positive rational radii.\\[1em]
Claim that $F\subset \mathcal{T}$ is the countable base required.\\[1em]
To do so the following observations are made:
\begin{itemize}
	\item Every real number $r$ can be written as a Cauchy sequence of $r_i\in \mathbb{Q}$ such that $r_i \nearrow r$
	\item Similarly, every $x$ can be written as a Cauchy sequences of $x_i\in \mathbb{Q}^n $ such that $x_i \nearrow x$
	\item Then consider the ball $B_r(x)$ where $r\in (0,\infty), x\in \mathbb{R}^n$ can be written as $B_r(x)=\bigcup\limits_{i=1}^{\infty}{\mathcal{U}_i}$ where $\mathcal{U}_i \in \mathcal{F}$
\end{itemize}
However, there is a problem, sets in $\mathcal{T}$ can be uncountable unions, 
meaning it is unknown whether the union, as above, is countable or not.\\[1em] 
To resolve this and to avoid an uncountable union, requires the fact that 
$\mathbb{Q}^n$ is dense in $\mathbb{R}^n$.\\[1em]
Pick an arbitrary $p\in \mathbb{R}^n$ and an open set$ \mathcal{U}_0 \in 
\mathcal{T}$ such that $p\in \mathcal{U}_0$.\\[1em]
The Hausdorff property of $\mathbb{R}^n$ means that there exists an $\varepsilon$ such that $B_\varepsilon(p) \subset \mathcal{U}_0$.\\[1em]
The density of $\mathbb{Q}$ means that one can take a $\delta$ such that $\delta \in  (\frac{\varepsilon}{2},\varepsilon) \cap \mathbb{Q}$\\[1em]
Since a rational centre is needed also, take a $q \in \mathbb{Q}^n$ such that $d(p,q)<\displaystyle\frac{\varepsilon}{8}$ (using the density of $\mathbb{Q}^n$).
Then $B_\delta(q) \in F$ and $p\in B_\delta(q)\subset \mathcal{U}_0$, and since 
the choice of $p$ was arbitrary, this completes the proof.
\subsubsection{4. Construction of the Atlas}
Recall that an atlas is a set $\mathcal{A} = \left\{ x_\alpha: 
\mathcal{U}_\alpha \rightarrow \mathbb{R}^n \text{ where } \alpha \in 
A,\mathcal{U}_\alpha \in \mathcal{T},  \bigcup\limits_{\alpha \in 
A}{\mathcal{U}_\alpha}=\mathcal{M}^n,\text{ and } x_\alpha\text{ is injective} 
\right\}$ and for each $x_{\alpha}, x_{\beta} \in A$, the map $x_\alpha \circ 
x_\beta^{-1} : x_\beta(\mathcal{U}_\alpha \cap \mathcal{U}_\beta) \rightarrow 
x_\alpha(\mathcal{U}_\alpha \cap \mathcal{U}_\beta)$ is a smooth 
homeomorphism\\[1.5em]
Let $x_0: \mathbb{R}^n \rightarrow \mathbb{R}^n$ be given by $x_0(x)=x$, that is the identity map, and claim that $A=\{x_0\}$ is an atlas.\\[1em]
Claim that $x_0$ is a chart. Its domain is $\mathbb{R}^n \in \mathcal{T}$ and 
the target is $\mathbb{R}^n$ and $\mathbb{R}^n$ covers $\mathbb{R}^n$.\\[1em]
It is also injective as $x_0(x)=x_0(y)\Rightarrow x=y$ (it is therefore a 
bijection)\\[1em]
Claim that the transistion maps are homeomorphisms. Let $\tau = x_0 \circ 
x_0^{-1}$. Then $\tau: \mathbb{R}^n \rightarrow \mathbb{R}^n$ is simply the map 
$\tau(x)$ which is the identity map and is a smooth (analytic) homeomorphism.
\subsubsection{5. Conclusion}
It is possible to conclude that $(\mathbb{R}^n, \mathcal{T}, \mathcal{A})$ is a smooth manifold where $\mathcal{T}$ is the topology generated by the distance function (or the notion of the 'open ball') and $\mathcal{A}=\{x \mapsto x \}$\\[1em]
\textit{Discussion:} Are there other charts that could be used to form the atlas for $\mathbb{R}^n$?
\newpage\noindent
\begin{example}[$\mathbb{S}^1$ as a manifold]
The manifold $\mathbb{S}$ is simply the circle, it can be defined $\mathbb{S}^1 = \{x \in \mathbb{R}^2: |x|=1 \}$\\
\vpi
\bluetxt{2.1 Set the Topology:}\\[1em]
Let $\mathcal{T}^{(\mathbb{S}^1)} = \{\mathcal{U} \subset \mathbb{S}^1: \mathcal{U}=\mathcal{U}_0 \cap \mathbb{S}^1,\text{ where } \mathcal{U}_0 \in \mathcal{T}^{(\mathbb{R}^2)} \}$.\\
\textit{Remark:} This is called the \textem{subspace topology}\\
\vpi
\bluetxt{2.2A Check the topology is Hausdorff:}\\[1em]
Claim that the desired property of the topology is inherited from $\mathbb{R}^2$. So let $p, q \in \mathbb{S}^1$ with $p \ne q$.\\[1em]
Since $\mathbb{R}^2$ is Hausdorff, there exists $\mathcal{U}_p,\; \mathcal{U}_q$ such that $\mathcal{U}_p,\;\mathcal{U}_q \in \mathcal{T}^{(\mathbb{R}^2)}$, and $p \in \mathcal{U}_p,\; q \in \mathcal{U}_q,\; \mathcal{U}_p\cap\,\mathcal{U}_q=\emptyset.$\\[1em]
Let $\mathcal{V}_p =\mathcal{U}_p\cap\,\mathbb{S}^1 \in \mathcal{T}^{(\mathbb{S}^1)},\; \mathcal{V}_q =\mathcal{U}_q\cap\,\mathbb{S}^1 \in \mathcal{T}^{(\mathbb{S}^1)}$ and $p\in \mathcal{V}_p, q\in \mathcal{V}_q,$ then $ \mathcal{V}_p \cap\,\mathcal{V}_q = \emptyset$ as required.\\[1em]
\textit{Discussion:} Does "inherited" make sense here?
\newpage \noindent
\bluetxt{2.2B Check the topology is Second Countable:}\\[1em]
Claim that the desired property is inherited from $\mathbb{R}^2$.\\[1em]
Let $\mathcal{F}^{(\mathbb{S}^1)}= \left\{\mathcal{V}: \mathcal{V} = \mathcal{U}\cap\,\mathbb{S}^1 \text{ where } \mathcal{U} \in \mathcal{F}^{(\mathbb{R}^2)} \right\}$, where $\mathcal{F}^{(\mathbb{R}^2)}$ is the countable base in $\mathbb{R}^2$.\\[1em]
Since $\mathcal{F}^{(\mathbb{S}^1)} \subset \mathcal{F}^{(\mathbb{R}^2)}$, it is countable. Now it remains to check that $\mathcal{F}^{(\mathbb{S}^1)}$ is a base.\\[1em]
Let $p\in \mathbb{S}^1,\; \mathcal{V}_0 \in \mathcal{T}$ such that $p\in \mathcal{V}_0$. Since $\mathcal{V}_0 \in \mathcal{T}^{(\mathbb{S}^1)}, \mathcal{V}_0 = \mathcal{U}_0 \cap\, \mathbb{S}^1$ , where $U_0 \in \mathcal{T}^{(\mathbb{R}^2)}$.\\[1em]
Then since $\mathcal{F}^{(\mathbb{R}^2)}$ is a base, there exists a $\mathcal{U}_1 \subset \mathcal{U}_0$ such that $p \in \mathcal{U}_1$ and $ \mathcal{U}_1 \in \mathcal{F}^{(\mathbb{R}^2)} \subset \mathcal{T}^{(\mathbb{R}^2)}$.\\[1em]
Take $\mathcal{V}_1 = \mathcal{U}_1 \cap\, \mathbb{S}^1$. Then $p\in \mathcal{V}_1 \in \mathcal{F}^{(\mathbb{S}^1)}$, where $\mathcal{V}_1 \subset \mathcal{V}_0$ as required.\\
\vpi
\bluetxt{2.3 Construction of the Atlas:}\\[1em]



\end{document}
% [/body]
% [/html]